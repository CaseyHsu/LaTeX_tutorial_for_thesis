\documentclass[class=NCU_thesis, crop=false]{standalone}
% \let\orilabel\label % enable \ref in LTXexample
\begin{document}

% \begin{LTXexample}[style=LatexStyle,xleftmargin=0.5cm]
% \end{LTXexample}
% \begin{lstlisting}
% \end{lstlisting}
% \textbackslash{}

\section{封面製作方法}

製作封面的方法在常見的 \LaTeX 教學中,僅提到利用 \textbackslash{}maketitle 指令可以產生封面頁。但使用後會發現其產生的封面極其簡陋,往往不合我們的需求。這裡要介紹的就是產生自訂封面的基礎方法。


在開始之前,讓我們靜下心想想,封面與內文的差異處:
\begin{enumerate}
    \item 不顯示、計算頁碼。
    \item 字體大小、位置、對齊方式等等都是自由可變化的。
\end{enumerate}

第一個要求可以透過 titlepage 環境達成,其作用是表明這是封面(未必是首頁),同時去除頁面的多欄(columns)設定、停用頁碼計數、修改頁面風格、另起新頁等等。
\footnote{\textbackslash{}maketitle 指令其實就是利用 titlepage 定義一個非常簡單的格式,請參考 book.cls }

第二個字體部份,與內文區大同小異。字的大小可用 \textbackslash{}Large 這類指令進行修改(或是\textbackslash{}fontsize\{size\}\{skip\})。對齊方式常用的如 center 環境來置中文字。文字行的間距除了可以用 Word 常用方法~直接換行外,\LaTeX 中更好的做法是利用\textbackslash{}vspace*\{\} 指令強制插入一個固定的垂直空白(須指定單位如 \vspace*{3mm} )。
\footnote{若使用不加星號的 \textbackslash{}vspace\{\} ,在頁首處將會失效,各位可斟酌使用。(\url{http://tex.stackexchange.com/q/76101})}
另外 \TeX 也提供了一個可變長度的填充指令 \textbackslash{}vfill ,\TeX 會將頁面文字剩餘空間平均分配給生成頁面上所有的 \textbackslash{}vfill 。舉例來說,以下範例會讓文字出現在1/3的位置。
\footnote{其實還有另一個指令 \textbackslash{}vfil 。\textbackslash{}vfil 與 \textbackslash{}vfill 作用相同但階層不同。\textbackslash{}vfill 較「高級」,能夠搶走\textbackslash{}vfil 的空間。詳見\url{http://tex.stackexchange.com/q/64756/111676}}
\begin{lstlisting}
\begin{titlepage}
\vfill
Here is 1/3 position.
\vfill
\vfill
\end{titlepage}
\end{lstlisting}

除了上面的基本格式方法外,還有幾個小地方要注意,這裡我直接用我的樣板設定來講。
\begin{lstlisting}
\begin{document}

\titlepageFontFamily
\newgeometry{top=2.0cm, bottom=2.0cm, inner=2cm, outer=2cm}
\begin{spacing}{1.0}
    \begin{titlepage}
        \null\vfill
        \begin{center}
        Text.
        \end{center}
    \end{titlepage}
\end{spacing}
\restoregeometry
\normalfont % use main font
\cleardoublepage 
\end{lstlisting}
其中要注意的指令及作用如下表:
\begin{table}[!h]
\begin{tabularx}{\textwidth}{lX}
\textbackslash{}titlepageFontFamily     & 自訂指令,為了設定封面頁的字型風格,如 \textbackslash{}sffamily \\
\textbackslash{}newgeometry\{\}         & 使用一個新的頁面規劃(頁邊寬度之類),並暫存之前的設定。這指令常用在封面規劃與內文不同的情況,尤其是左右邊距不同的情況。 \\
\textbackslash{}begin\{spacing\}\{1.0\} & 固定行距。因內文行距設定不應影響封面,所以這裡設定一個固定值。當然,若之前未更動行距設定,這裡也不用加。(spacing 環境需載入 setspace 套件) \\
\textbackslash{}restoregeometry         & 復原至上一個頁面規劃。\\
\textbackslash{}normalfont              & 復原至預設字型\\
\textbackslash{}cleardoublepage         & 強制換至右頁(用於雙面模式下,單面模式就單純換面)。\\
\end{tabularx}
\end{table}


上面這些就是使用純文字製作封面的方法了。若要仿造某個封面(如:某校提供的範例)或是做較為精細的版面設計,建議在\textbackslash{}documentclass 的選項加上 colorgrid ,可以顯示以 mm 為單位的網格用於對齊(由 eso-pic 套件提供)。如果想放入圖片、表格請直接用 \textbackslash{}includegraphics\{\} 與 tabular (毋須浮動環境figure、table)。若是要放入繪製圖、線段等等可以參考 tikz 繪圖。


% 要製作一個封面,須要使用到 titlepage 環境(用\begin{...} \end{...}包起來)與 \vfill 指令。
% titlepage 環境的作用是表明這是封面(未必是首頁),同時去除頁面的多欄(columns)設定、停用頁碼計數、修改頁面風格、另起新頁等等。(     \maketitle 指令其實就是利用 titlepage 定義一個非常簡單的格式)
% 
% \vfill 指令則是在指令位置插入一個可變大小的垂直距離(空白)。 LaTeX 會將頁面文字剩餘空間平均分配給生成頁面上所有的 \vfill 。舉例來說,以下範例會讓文字出現在1/3的位置。
% \begin{lstlisting}
% \begin{titlepage}
% \vfill
% Here is 1/3 position.
% \vfill
% \vfill
% \end{titlepage}
% \end{lstlisting}
% 
% 接著還有 center 環境與 \vspace*{…} 指令可以控制文字位置。center 環境就是將文字置中,而 \vspace*{…} 則是強制插入一個固定的垂直空白(須指定單位如 \vspace*{3mm} )。
% 稍微提一下,  若使用不加星號的 \vspace{…} ,在頁首處將會失效,各位可斟酌使用。(
% http://tex.stackexchange.com/q/76101       )
% 
% 進一步了解
% 除了上面的基本格式方法外,還有幾個小地方要注意,這裡我直接用我的樣板設定來講。
% 



\end{document}
% 
% \begin{titlepage}%
%   \let\footnotesize\small
%   \let\footnoterule\relax
%   \let \footnote \thanks
%   \null\vfil
%   \vskip 60\p@
%   \begin{center}%
%     {\LARGE \@title \par}%
%     \vskip 3em%
%     {\large
%      \lineskip .75em%
%       \begin{tabular}[t]{c}%
%         \@author
%       \end{tabular}\par}%
%       \vskip 1.5em%
%     {\large \@date \par}%       % Set date in \large size.
%   \end{center}\par
%   \@thanks
%   \vfil\null
%   \end{titlepage}%
%   \setcounter{footnote}{0}%
%   \global\let\thanks\relax
%   \global\let\maketitle\relax
%   \global\let\@thanks\@empty
%   \global\let\@author\@empty
%   \global\let\@date\@empty
%   \global\let\@title\@empty
%   \global\let\title\relax
%   \global\let\author\relax
%   \global\let\date\relax
%   \global\let\and\relax
