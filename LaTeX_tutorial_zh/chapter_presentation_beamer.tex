\documentclass[class=NCU_thesis, crop=false]{standalone}
% \let\orilabel\label % enable \ref in LTXexample

\begin{document}

% \begin{LTXexample}[style=LatexStyle,xleftmargin=0.5cm]
% \end{LTXexample}
% \begin{lstlisting}
% \end{lstlisting}
\let\tnote\tptabletnote
\chapter{用 Beamer 做簡報}
\LaTeX 除了排版書籍,製作簡報也是沒問題的。如果你厭倦了 GUI 介面的簡報軟體如 M\$ Office 的 PPT 或是 Libreoffice 的 impress 須要精密的滑鼠操控技術以及多圖下 Lag 到不行的體驗,那\LaTeX 簡報或許是你的救贖。不過對於常常須要建立箭頭指標等等的簡報類型,可能會讓你花費更多的時間,請依個人情況決定是否採用。

其實政大數學系的蔡炎龍老師做過一份淺顯易懂的\href{http://yenlung.km.nccu.edu.tw/xms/read_attach.php?id=85}{簡報}。
\footnote{由\url{http://yenlung.km.nccu.edu.tw/xms/read_attach.php?id=85}下載。}
下面的內容算是自己的心得整理。

\section{快速上手}

\LaTeX 做簡報有很多種方法,其中最流行的就是 Beamer 。
\footnote{Beamer 無論是\LaTeX 、 PDFLaTeX 或是 \XeLaTeX 都支援。其他製作方法有些限定了可用的編譯器,選用上須注意。}
一份最簡單的(英文) Beamer 簡報如下:
\begin{lstlisting}
\documentclass[xetex,mathserif,serif]{beamer}
% \usetheme{Warsaw}

\title{Title of Presentation}
\subtitle{Subtitle of Presentation}
\author{sppmg}
% \institute{}
% \date{}

\begin{document}
\frame{\titlepage}

\begin{frame}{Title of slide}
Content goes here
\end{frame}

\begin{frame} %[plain], [shrink], [allowframebreaks], [fragile]
\frametitle{Title of slide}
\framesubtitle{Subtitle of slide}
Content goes here
\end{frame}

\end{document}
\end{lstlisting}
這個例子將會生成3頁的簡報。各項解釋如下:
\renewcommand*{\thead}[1]{\multicolumn{1}{|c|}{\bfseries #1}}
\begin{table}[h]
    \begin{threeparttable}
         \caption{}
        % --------------------------
        \begin{tabularx}{\textwidth}{|l|X|} \hline
\thead{行號或項目}   & \thead{解說} \\ \hline
1     & 使用 beamer class 。選項使用 xetex 是因為我使用 \XeLaTeX ,非 \XeLaTeX 使用者請依手冊修正。 \\ \hline
2     & 選擇簡報主題。 \tnote{1} \\ \hline
4 $\sim$ 8  & 設定簡報題目、作者資訊 \\ \hline
11    & 產生封面 \\ \hline
\textbackslash{}begin\{frame\} $\sim$ \textbackslash{}end\{frame\} & 一頁簡報。 \\ \hline
\textbackslash{}frametitle\{\}     & 各頁標題,亦可直接以\{\}加在\textbackslash{}begin\{frame\}後(第1參數)。 \\ \hline
\textbackslash{}framesubtitle\{\}   & 各頁子標題,亦可直接以\{\}加在\textbackslash{}begin\{frame\}後(第2參數)。 \\ \hline
frame選項 & 擇一使用或不用亦可 \\ \hline
plain               & (經測試,有些頁面下方的附加資訊會消失,可略增頁面空間。) \\ \hline
shrink              & 將一頁內容縮小至符合頁面大小。可用[shrink=5]來設定允許縮小比例。若內容小於頁面將向上對齊而非至中。 \\ \hline
allowframebreaks    & 若超過頁面的話可以自動分成另一頁(頁標題自動附加數目)。 \\ \hline
fragile             & 若需verb 、 verbatim 、 lstlisting 等原樣輸出指令,須用此選項。 \\ \hline
%         http://nochair.net/posts/2011/05-05-fragile-latex-beamer.html
        \end{tabularx}
        % --------------------------
        \begin{tablenotes}
            \item [1] 完整主題截圖列表請見:\url{https://www.hartwork.org/beamer-theme-matrix/} 。 縱軸為主題,橫軸為色彩(\textbackslash{}usecolortheme\{\})。
        \end{tablenotes}
    \end{threeparttable}
\end{table}

\section{文字強調及色彩}
一般 \LaTeX 文件中常用 \verb|\emph{}| 做\emph{強調文字},但這指令在 Beamer 中會太斜,也不夠醒目。因此 Beamer 中另提供\verb|\structure{}|與\verb|alert{}| ,\verb|\structure{}|通常為粗體,顏色則視主題而定。
\verb|alert{}|一般使用{\color{red}紅色}文字。

若要使用其他顏色可以用 \verb|\color{color}| ,記得加上\{\}限制範圍(這方法與一般文件是一樣的)。
\begin{LTXexample}[style=LatexStyle,xleftmargin=0.5cm]
This is {\color{green} green color} word.
\end{LTXexample}

\section{文字方塊}
Beamer 提供文字方塊,如 。但須注意預設的主題中並未定義各方塊樣式,請用\verb|\usetheme{}|更改主題。



\section{分段顯示}
這裡的「分段顯示」指的是在逐步切換過程中,一頁的文字會逐漸顯示出來。譬如說一個列表每按一次顯示一個項目。這個功能在 Beamer 中稱為 ``overlay'' 。

\section{縮短編譯時間}
% Improving Compilation Speed
% \documentclass[draft]{beamer}
% \includeonlyframes{⟨frame label list⟩}
縮短編譯時間有兩種方法。一是使用 draft 選項,如同一般文件,draft 將不處理圖片及目錄。
\begin{LTXexample}[style=LatexStyle,xleftmargin=0.5cm]
\documentclass[draft]{beamer}
\end{LTXexample}

另一種方法類似一般文件的\verb|\includeonly{}|,僅編譯所標記下來的頁面。(但似乎無法配合 allowframebreaks )
\begin{LTXexample}[style=LatexStyle,xleftmargin=0.5cm]
\includeonlyframes{current}
\begin{frame}[label=current]{title}
...
\end{frame}
\end{LTXexample}


\end{document}